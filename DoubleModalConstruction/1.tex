% !TeX program = xelatex
\documentclass[fontset=ubuntu]{ctexart}

\usepackage[a4paper]{geometry}

\usepackage[colorlinks, allcolors=blue]{hyperref}
%\usepackage{footmisc} % make \footnote link not work
\usepackage{footnotebackref} % must behind `hyperref` and `footmisc`

\usepackage{amssymb, amsmath}

\usepackage{natbib}
\usepackage[sort&compress]{gbt7714}
%\bibliographystyle{gbt7714-numerical}
\bibliographystyle{gbt7714-author-year}

\usepackage{covington}

\renewcommand\refname{参见}

\newcommand{\textul}[1]{\makebox{\underline{#1}}}
%\newcommand{\footnoteref}[1]{\textsuperscript{\normalfont\ref{#1}}}

\begin{document}

	\title{模态逻辑简介与双重模态词的逻辑分析}
	\author{Velt of Alfeberg\footnote{E-mail: jiyu8iighx@outlook.com}}

	\maketitle
	
	\section*{标签}
		\begin{itemize}
			\item 撰文目的/介绍
			\item 先修知识/命题逻辑
			\item 研究对象/元研究,设计对象的现象与一般规律
			\item 方法/建模,案例分析
			\item 现实与模型/建模
			\item 经典领域划分/模态逻辑
			\item 许可证/CC-BY 4.0
			\item 语义/概念分析
			\item 语义/概念/模糊性,可能性/模态
		\end{itemize}
	\section*{摘要}
		本文介绍模态逻辑公理与关系语义框架,并在这些理论的基础上分析双重模态词的使用。
		并讨论此类研究对人造语言设计者的意义、此类设计对用户的意义。

	\newpage

	\section{模态逻辑}

		「模态词」是「模态」的语言形式表示。
		「模态」包含「认识模态」、「真势模态/虚拟可能性」、「道义可能性」等,涉及多种较为不同的语言和思维过程,
		它们之所以归为一类\footnote{已知是由亚里士多德和一些可能更早的哲学家最先讨论的,现代的分类动机可能有所转变而与古时不同。},主要是因为它们具有类似的逻辑形式。

		\subsection{逻辑上的模态算子}

			通常有一个基本的模态算子,称为\textbf{必然算子}而记作\(\square\),它与否定的复合会构成另外三个衍生的模态算子。实际上在这四个算子中任取一个都可以定义出其余三个\footnote{也有一些模态逻辑教科书将此性质不成立的系统作为讨论内容,如\citet{cresswell2012new}。}。常提及的\textbf{可能算子}记作\(\Diamond\),定义为
			\begin{equation}
				\Diamond A := \lnot\square\lnot A
			\end{equation}
			其中\(\lnot\)是逻辑否定,这性质也称为必然与可能间的\textbf{对偶律}(德摩根律)。

			逻辑算子要成为模态算子,可能还需要满足一系列逻辑公理和推理规则。模态逻辑普遍要求除肯定前件外,还要有\textbf{必然性规则}:
			\begin{equation}
				\frac{\vdash A}{\vdash\square A}
			\end{equation}
			即:如果一个命题是(无需额外前提的)定理,则它的必然性也是定理。对于一个逻辑系统,这意味着所有的重言式都是必然的(逻辑可能性)/已知的(认识可能性)。
			
			公理体系则有多种。满足\textbf{分配公理}(实际上是公理模式。此公理模式也称为\textbf{克里普克模式})
			\begin{equation}\label{modal_dist_axiom}
				\square(A\to B)\to(\square A\to\square B)
			\end{equation}
			的公理体系称为\textbf{正规模态逻辑}。

			在各种正规模态逻辑体系中,\(K\)是最基本的,它只包含以上提到的这些要求。而对于不同含义的模态算子,可能需要各种各样不同的公理来刻画。如体系\(T\)增加了一条公理\(T\)作为要求\footnote{一些作者也将其记作体系\(M\)、公理\(M\)。}:
			\begin{equation}
				\square A\to A
			\end{equation}

			以上基于逻辑公式的定义是纯形式的,根据具体语义的区分模态算子又可分为多种。常见的有:

			\begin{enumerate}
				\item 逻辑可能性。
				
					因为我没见过喷火龙,所以我推测:喷火龙的存在是\textul{不可能}的,进而喷火龙的不存在是\textul{必然}的,故喷火龙不存在。

				\item 时间可能性。
				
					由于糖是甜的,从而糖在\textul{未来总是}甜的、在\textul{过去总是}甜的。

				\item 道义逻辑。
				
				因为犯罪行为破坏了高集体利益的博弈均衡,所以我认为:犯罪是\textul{禁止}的,也就是说不犯罪是\textul{应当}的、犯罪不是\textul{允许}的。

				\item 信念逻辑。
				
					对于一个准确的推理者,如果明天不下雨,那么他\textul{不相信}明天下雨;对于一个自恰的推理者,如果他\textul{相信}明天下雨,那么他\textul{不相信}明天不下雨。\footnote{两个推理者的例子分别对应于模态逻辑公理\(T\)和\(D\)。}
			\end{enumerate}

			对于例1中的逻辑可能性,推出「喷火龙不存在」相当于是使用了公理\(T\)。而此公理对于道义逻辑则未必成立,否则我们就有「由于某人不犯罪是应当的,那么某人不犯罪」——这显然是道德高尚之人的情况、而不是描述失德之人乃至整个非理想社会时所适合用的可能性观念。

			这一区别对于信念逻辑更为明显:可以有各种各样的推理者,他们的信念适用的公理体系不同。\footnote{但道义逻辑和信念逻辑不一定是正规模态逻辑。}

		\subsection{模态算子的形式语义}

			模态算子没有经典逻辑意义上的外延语义,也就是说:包含模态词的句子的真值不能由其组分的真值确定。「我活着」和「2+2=4」同样是真的,但「我必然活着」为假而「2+2=4必然成立」为真——显然模态算子无法是一个真值函数。一般来说,真值语义的失效意味着以下二者中的一个或多个:
			
			\begin{itemize}
				\item 语义是内涵的:判断模态语句的真伪需要了解其组分命题的具体结构(内涵),而不仅仅是组分命题的真值。
				\item 语义是语境的:判断模态语句的真伪需要语句外的信息。
			\end{itemize}

			对此,模态逻辑有一个著名的语义形式化方案:\textbf{关系语义}(也称克里普克语义)。
			现在我们需在一个元语言讨论问题,对象语言中的任何命题(包括模态命题)\(A\)应转译到这个元语言中表达为\(w\Vdash A\),其中\(w\)称为\textbf{当前世界},二元关系\(\Vdash\)称为\textbf{满足关系}。
			
			此外,有以下经典逻辑风格的公理:
			\begin{itemize}
				\item 排中 \begin{equation}
					w\Vdash\lnot_\mathrm{obj} A\quad\longleftrightarrow\quad w \nVdash A
				\end{equation}
				\item 实质条件 \begin{equation}
					w\Vdash A\underset{\mathrm{obj}}{\longrightarrow} B\quad\longleftrightarrow\quad (w\nVdash A\lor w\Vdash B)
				\end{equation}
			\end{itemize}
			其中\(w\nVdash A:=\lnot(w\Vdash A)\),而\(\mathrm{obj}\)则用于将对象语言中的逻辑符号同元语言中的区分开来。

			我们最为关心的是以下公理在元语言层面所完成的对模态算子的定义:
			\begin{equation}
				w\Vdash \square A \quad\longleftrightarrow\quad \forall u(wRu\to u\Vdash A)
			\end{equation}			
			其中二元关系\(R\)称为\textbf{可及关系}。
			
			这些\(\Vdash\)前方的、\(R\)两端的或被元语言量词所指涉的变量(如上文公式中出现的\(u,w\)),称为\textbf{世界}。从这些公理出发进行演绎推理,不会产生如\(\forall A(A\Vdash u)\)或\(R(A,B)\)这样的\textit{病态公式}。也就是说:若公理模式的一些实例(即,公理)不是病态的,则它们的演绎闭包也不是病态的。故无需在形式系统内额外对这两个关系施加类型约束,「世界」与「世界所满足的命题」之间的区分是不言自明的。

			现在考虑模型论风格的形式语义,若指明「世界」所构成的集合(量词指涉的论域),则各关系的结构乃至模态算子的语义也将确定,此即关系语义。

			模态算子的诠释原本是无法在对象语言中不依赖语境地完成的,而现在我们在元语言中引入了可及关系和满足关系,通过额外地、显式地要求世界上的可及关系结构的知识以及世界与命题间的满足关系结构的知识(或者说「世界是由怎样的一系列命题构成」),语义不再是语境的。而对于简单的、尤其是有限的论域,量词以及两个关系的解释可以仅仅是真值函数。

			不过,以上定义要成为模态逻辑的语义,还需蕴含模态逻辑公理。可以证明%按定义展开为三个式子的析取,它们无法同时为假
			,在此框架内分配公理(式\ref{modal_dist_axiom})恒成立,从而关系语义所建模的只能是正规模态逻辑。关系语义的另一个优势是,许多正规模态逻辑体系中的公理都可由一些通用的(即,无论如何选取其中量词的论域都有效)可及关系的结构蕴含。如上文提到的公理\(T\)在关系语义中成立的一个充分条件\footnote{不是必要条件,但在要求这个条件仅是\(R\)的性质且不含有任何自由变量的情况下,自反性是最简单的充分条件。}是可及关系的\textbf{自反性}
			\begin{equation}
				\forall x, xRx
			\end{equation}
			这样的可及关系的性质与模态逻辑公理之间的关系,使得这套建模方案具有较好的解释性,后文将有例子(式\ref{eq:modal_kripke_trans})体现这一点。

			对于生活中常见的可能性,关系语义还具有很好的解释性。比如,下列语句中的模态词若使用关系语义并视为知识可能性来诠释,得到的结果并不比日常语言的直接表达难理解。
			
			\begin{covexamples}
				\item 
					\gll 厦门 不 可能 在非洲
						\(x\) \(\lnot\) \(\Diamond\) \(F\)
					\glt \(\lnot\Diamond F(x)\)
					\glend
				\item 
					\gll 不 存在一个 兼容于 当前知识状态 的 可能知识状态 , 其 包含 「 厦门 在非洲 」
						\(\lnot\) \(\exists\) \(R\) \(w\) {} \(u\) {} \(u\) \(\Vdash\) {} \(x\) \(F\) {}
					\glt \(\lnot\exists u,\quad wRu\land u\Vdash F(x)\)
					\glend
			\end{covexamples}

			关系语义将模态算子表达为量词的另一个优势是,允许通过将量词推广为广义量词(如半定量量词)来表达更多类型的可能性(如「很有可能」)。

		\section{自然语言中的双重模态词}
			
			\textbf{双重模态构造}(Double Modal Construction)乃至多重模态构造是一类普遍存在但相对不常见的表达形式,一个著名的中文案例即「\textul{大约}孔乙己\textul{的确}死了」\footnote{语出\citet{魯迅1923吶喊}所撰短篇小说《孔乙己》。},其在一个主谓结构中出现了两个模态词。

			这类语言现象存在于英语(美国南部)、德语、荷兰语、汉语等自然语言中,但这些句子的含义通常有分歧,缺乏共识甚至存在关于合法性的争议,不同人对同一语言的同一表达的接受程度也有不同,也有显著的地域特征。在这类现象的文法研究中,会较强烈地体现「描述文法」与「规范文法」这两种研究对象间的差异。
			
			模态词的相对顺序以及与词性/句法成分系统的兼容行为是如何依赖于含义的?
			由于自然语言缺乏明确、严整的规则,试图基于语义来对双重模态词的语法现象进行分析是比较艰难的任务。
			如:英语中的模态词「could」的使用需求还可能源于对祈使的婉约性的追求——这类用法已经较大程度上偏离了可能性这一话题;「must」、「need」、「have to」等英语模态词的词性也不相同,分析起来较为复杂。
			这可能意味着,自然语言的这一现象的研究方法应当是逻辑、认知、心理的先行,在此基础上再进行统计、归纳研究。

			\subsection{并列的模态词}

				并列的同类可能性的表达通常是不需要的,因为较“弱”的可能性会蕴含于较“强”的那个。在「我也许大概关了窗户」中,如果认为两个模态词是合取的关系——也就是说相当于「我也许关了窗户且我大概关了窗户」,且认为「大概」所表达的「关了窗户」的概率高于「也许」,那么「大概」显然就蕴含了「也许」,从而后者是多余的。这也是为什么读者通常认为这句话是古怪的。

				而「大约孔乙己的确死了」在其语境中则显得没有那么奇怪,甚至富有文学韵味。这可能是由于这两个强度不同的可能性的推理依据不同,而作者不想抹消它们的存在与它们之间的差异。也就是说,作者所表达的是「根据我的设想G,孔乙己可能死了;根据我的设想H,孔乙己多半是死了」,但考虑到表达成本和艺术性从而没有直白地写出。

				从这句话的上下文看来,「可能死了」的推断是来源于「许久没见到孔乙己了,如果说是他死了就说得通了」,而「多半死了」的推断是源于「他腿断了又好喝懒做又行窃招打,想必是活不下去」。
				
				值得注意的是,这两个模态词在句子中出现的顺序是符合认识顺序的:“我”先是提出「孔乙己死了」的猜想来作为「许久没见到孔乙己了」的解释,这样的解释是充分但非必要的,从而适用一个较弱的模态词「大约」;为了进一步佐证这个结论,“我”想到了「他腿断了又好喝懒做又行窃招打」,这个条件足以较为可靠地证明「孔乙己之死的必然性」,且与前面的推理相印证,故使用了模态词「的确」以表\textul{高可能性}和\textul{确认}。

				综上所述,同类可能性的模态词所构成的句子间往往存在蕴含关系,通常不需要同时使用两个。使用有蕴含关系的两个模态词的一个动机在于:考虑到人读到一句话时所进行推理量往往是较为有限的,如果有其他的“言外之意”需要读者来思考,那么最好尽量把“言内之意”充分多地呈现出来以作为思考的引子。
				另一个动因则在于表达者的认识过程细节,如口语中对多重模态词的使用可能是「边说边想从而需要根据新的思考结果来对可能性作修正」的结果。

				具体的语言心理、认识过程较为复杂,通常不是逻辑学的研究内容,但句子本身可以有简单的形式化方案\footnote{本例实际上应该用半定量量词来形式化,不当之处被忽略是出于本文的写作目的。}:

				\begin{covexamples}
					\item 
						\glosspreamble{模态算子\newline}
						\gll 大约 孔乙己 的确 死了
							\(\Diamond_G\) \(k\) \(\square_H\) \(D\)
						\glt \(\Diamond_G D(k)\land\square_H D(k)\)
						\glend
					\item 
						\glosspreamble{关系语义\newline}
						\gll 在 当前认识状态(包含「许久没见到孔乙己了」这一观察) 的各种可能解释中 存在(一定数量的) 包含着 孔乙己 死了 的
							{} \(w_G\) \(R_G\) \(\exists\) \(\Vdash\) \(k\) \(D\) {}
						\glt \(\exists u,R_G(w,u)\to u\Vdash D(k)\)
						%\glt \((\exists u,R_G(w,u)\to u\Vdash D(k))\land(\forall v,R_H(w,v)\to v\Vdash D(k))\)
						\glend
						\gll 在 当前认识状态(包含「他腿断了又好喝懒做又行窃招打」这一记忆) 的各种可能结果中 (几乎)没有不 满足 孔乙己 死了 的。
							{} \(w_H\) \(R_H\) \(\nexists\lnot\)(即\(\forall\)) \(\Vdash\) \(k\) \(D\) {}
						\glt \(\forall v,R_H(w_H,u)\to v\Vdash D(k)\)
						\glend
						
				\end{covexamples}

				如果不使用下标\(G,H\)区分模态算子,那么就可以适用模态逻辑公理\(D\)\footnote{这个公理对自然语言中的绝大多数可能性而言都是恒成立的}:
				\begin{equation}
					\square P\to\Diamond P
				\end{equation}
				从一个可能性命题推出另一个。正因为作者的表达意图要求这个前提为假,从而双重模态词的使用将是不可避免的。

			\subsection{序连的模态词}

				除合取外,模态算子还可能是嵌套的。也就是说,我们可以把「大约孔乙己的确死了」理解为:

				\begin{covexamples}
					\item 
						\gll 大约 孔乙己 的确 死了
							\(\Diamond\) \(k\) \(\square\) \(D\)
						\glt \(\square\Diamond D(k)\)
						\glend
					\item
						\gll 存在一个 过去自己所知的孔乙己 所能进入的 可能世界(境遇) , 这个可能世界 的 所有 未来 可能世界 中, 孔乙己 都 死了
							\(\exists\) \(w\) \(R\) \(u\) {} \(u\) {} \(\forall\) \(R\) \(v\) {} \(k\) {} \(D\)
						\glt \(\exists u(R(w,u)\to (\forall v(R(u,v)\to v\Vdash D(k))))\)
						\glend
				\end{covexamples}

				简单地说就是「孔乙己大约是走到必死的境地了」。这样的表达的用意一般是为了引导读者去意识到这样一个「必死的境地」的存在性与现实合理性——尤其是对于鲁迅这样的尤为重视对社会中个体的命运的必然性的原因(民族劣根性)的反思的作者。

				这样的表达需求中多数日常生活场景中不存在,我们多数情况下只想表达「孔乙己可能死了」而已。
				直觉上显然的是,「孔乙己可能死了」这个含义可以蕴含于「大约孔乙己的确死了」。
				实际上,只要可及关系\(R\)满足传递性\footnote{传递性也将保证模态逻辑公理\(4\)的成立}
				\begin{equation}\label{eq:modal_kripke_trans}
					\forall x,y,z,\quad xRy\land yRz\to xRz,
				\end{equation}
				就很容易证明这样一个对任意命题\(A\)成立的定理:
				\begin{equation}
					\Diamond\square A\to\Diamond A
				\end{equation}
				也就是说,当可及关系的传递性不成立,很可能就意味着这个可能性的上述蕴含关系不成立。不过这并不意味着这种情况下双重模态词的使用会比较多,因为考虑到所有的非传递可及关系的连通性都可以作为一个传递的可及关系,表达者完全可以换用这种新的可能性——\underline{只要表达需求允许}。

			\subsection{显著不同的模态词}

				从上文中可以看出,双重模态词不能被单个模态词取代的一个较根本的原因在于两个模态词所对应的可能性实际上并不算是同一类:它们可能有不同的可及关系诠释、当前世界选取,甚至论域都大有不同。

				在自然语言语料中,出现得更多的是那些具有更大差异的双重模态词,如「我 \textul{可能} \textul{应该} 阻止他的」这句话中的「应该」是道义意义上的,与其中的「可能」一词有更为明显的不同。它们所遵循的形式规律(即,模态逻辑公理体系)往往是不同的,更不论可及关系诠释、当前世界选取这些细节的差异了。

			\subsection{模态逻辑以外的含义}

				双重模态词的含义有时不能用模态逻辑解释。如许多人用到多重模态词只是出于一种相对缺乏深思熟虑的需求:如果在句子中多加了一个「也许」,就能令句子听上去具有更低的置信度,从而表达者可以对自己的话语负有更少的责任。对这类现象的分析,以及对上文分析过的现象的深入讨论,必然会走向认识论。语义学家中也有一些曾试图提出近似的模型,并用于讨论“认识距离”(或者说“认识势垒”)等现象(实际上前文的关于非传递的可及关系的讨论正是对此问题的一个定性分析)。

		\section{关于人造语言与双重模态词}
			
			双重模态词的使用在自然语言中是未被普遍接受的,但上文的分析已充分说明它可以对应着许多实际存在的表达需求。这些需求通常意味着对“精细表达”的需要,当我们只需要“粗略表达”时,完全可以使用其所蕴含的一个只包含单个模态词的句子。

			此即表达的模糊性问题。汉语的规则及其概念系统的局限性决定了无法用很低的表达成本来实现可能性的细节区分,对于人造逻辑语言,这是应当尽力避免的情况,但这并不意味着可以允许模糊表达变得很笨重甚至被禁止。首要的任务是对可能性进行认识论层面的分析,构建精细的概念系统;然后是对语法进行优化,允许在限制表达成本的同时补充一些可选信息,保效率、保组合性乃至保认识次序;在这些概念与规则的设计完成后,可以多进行一些实用目的的含义指派(即词汇的命名等),消耗语言的形式的状态数(即可用的音节组合数等)、牺牲用户的学习时间来换取使用时的效率。
			
			对于一些信息博弈和艺术创作,恰到好处的模糊性往往是必要的,语言的特性将决定这种“语言游戏”的游戏规则。而模态词天生就是用于在精确的理想对象上引入模糊性与不可靠性的,处于这类问题的中心地位。

			正因为语言使用者、尤其是文学创作者能够根据自己的表达需求而突破习惯语法的桎梏,我们的语言的内涵才得以丰富,对自己的语言和思维的认识才能更加深入。这也反映了由数据(大量经验)驱动训练而来的「描述文法」的局限性,鼓励了语言设计者对「规范文法」的设计活动。

			此外,模态的逻辑研究也为各种可能性建立了法则性的描述。若将这些知识传授给用户,用户便能借助对法则的记忆来迅速而又可靠地进行涉及模态词的推理。这样的推理虽然一定程度上是未经审视的,将会树立新的教条,但这至少比在自然语言中树立不可靠的教条要好,而且退一步想,这些关于法则的知识还能用于筛选可靠的语言训练材料。目前的自然语言用户所得到的训练效果是:许多人连多重否定都需要废不少时间去思考其含义——更不要说多重模态词了,而当他们想要自行树立一个教条时,又会陷入复杂的自然语言分析的泥潭\footnote{关于复杂性,可从\citet{sep-negation}中的对自然语言否定的讨论中体会。}。当语言对于公共知识(尤其是其可靠性)有着更为深谋远虑的控制时,用户的思维便可以更多地解放出来,从而其他门类的工程性能指标的限制(如表达成本、模糊性)也将一定程度上得到缓解。
			
	\nocite{*}
	\bibliography{1.bib}

\end{document}
